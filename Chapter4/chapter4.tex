%!TEX root = ../thesis.tex
%*******************************************************************************
%****************************** Concluding Chapter *****************************
%*******************************************************************************

\chapter{Concluding remarks} %Title of last chapter
\label{chap:conclusion}

\ifpdf
    \graphicspath{{Chapter4/Figs/Raster/}{Chapter4/Figs/PDF/}{Chapter4/Figs/}}
\else
    \graphicspath{{Chapter4/Figs/Vector/}{Chapter4/Figs/}}
\fi

%********************************** %First Section  **************************************

This thesis has explored the causes and effects of several uncertainties in nuclear fusion reactor analysis and operation. It began by describing motivations for developing controlled nuclear fusion power, by situating new power generation technologies within a broader environmental and socio-political context. A growing world population demanding improving material living standards has increased global power demand to double its value of 40 years ago. Human-induced climate change, deteriorating air quality and a desire for greater energy security have helped spur the development of low-carbon electricity production technologies such as renewables, GEN-IV fission and fusion to meet contemporary and future demands for power.

Whilst the body of nuclear fusion knowledge has massively accumulated since the process was first discovered, constructing and operating a working fusion reactor, as opposed to an experiment, is still yet to happen. However we are now in the final stages of research and development before such a reactor is constructed and it is tremendously important to identify and quantify any risks to the realisation of controlled nuclear fusion as an economical, practical electricity generation scheme. Many of these risks are political and organisational, but as the introductory chapter alluded to, many are technical, engineering challenges. Hopefully none are physical impossibilities. 

Many of the technical risks are amplified by our inexact knowledge of them. A tremendous amount of work is currently undertaken to model the performance of current and future fusion devices, simulating performance and estimating parameters of interest. However, these parameters are always subject to some sort of uncertainty, whether it is quoted or not. Uncertainty in powers, Mean Time Between Failures (MTBF), particle fluxes, cost of electricity, Tritium Breeding Ratios and more. To decrease these uncertainties, to narrow their distribution around the most mean value, is to reduce the maximum potential risk associated with them and to give room for maneouvre in the trade-offs that come with engineering a functioning system. 

The work which comprises this thesis has investigated several sources of uncertainty of pertinence to the development of nuclear fusion as a power generation scheme. First, a stochastic, sampling technique known as Total Monte Carlo was used to explore the effects of nuclear data uncertainty in the TBR for a future fusion power plant design, DEMO. This technique has never before been used to estimate uncertainty on TBRs. Investigating the contribution of uncertainty from lead nuclear data, many radiation transport simulations of the DEMO device were performed, each tallying the TBR and sampling different lead nuclear data. This work used the TENDL2015 nuclear dataset, with fully correlated cross-channel behaviour for the reaction channels, angular distributions and other variables. The results of the work were to determine the standard deviation of the HCLL DEMO TBR due to lead, 1.2\% of the mean value. The simulated TBR distribution was not normally distributed, instead it had a negative skewness--a low-value tail. As a result, 5.8\% of the TBR distribution was less than unity. This only serves to reinforce the importance of higher-order moments in parameter probability distributions and the value of TMC style methods. Generally, where parameter mean values are close to the limits of some operational range, one should seek to know the shape of that parameter distribution, not just the extent. Whilst a TBR in a liquid-metal breeder blanket could potentially be tailored through on-line $^{6}$Li enrichment, this is not the case for ceramic type breeding concepts. For those, an overestimated TBR could be a costly mistake. After sampling the TBR distribution, the relationships between fundamental nuclear parameters and the TBR were investigated, with a handful of local Optical Model Potential (OMP) parameters responsible for most of the variation in TBR. In terms of future research, the determination of uncertainty contributions to TBR in lead blankets from other nuclides such as $^{16}$O, $^{56}$Fe and $^{nat.}$W would be a worthwhile effort. Advances in theory or new nuclear physics experimental data could help refine our models of lead nuclei and their behaviour, reducing uncertainties in lead based blanket designs.

The subsequent chapter looked at a particular modelling approximation, spatial homogenisation, where heterogeneity in material composition is artifically reduced by replacing many realistic materials with mass-conserving mixtures of materials. The effects of this approximation on radiation shielding were investigated. First the basic theory of radiation shielding was introduced, before a description of a comparative method for determining the discrepancy between heterogeneous and homogeneous modelling approaches. This method was employed to analyse how the approximation affects calculated dose rates for parameters relevant to the ITER tokamak. In one scenario, the on-load dose rate from D-T fusion was calculated on the far side of a reinforced concrete wall similar to the ITER bioshield. The discrepancy induced by the homogeneous approximation was a function of wall thickness, and attained a maximum of a 22\% underestimate of the neutron dose. This underestimate in the homogeneous simulation is due to the dispersed absorbing nuclei from the steel material which act as neutron sinks. There was less effect for photons, with a maximum underestimate of 10\%. The impacts of spatial homogenisation on the Shut-Down Dose Rate (SDDR) were also investigated. Activation at the internal face of the shield was found to be overestimated by a small amount by spatial homogenisation, approximately 10\%, although this did vary as a function of time since last irradiation. The effects of spatial homogenisation in breeding blankets for fusion have been explored by \citeauthor{Pelloni1989} amongst others \cite{Kumar1989}. However, these results pertaining to radiation shielding in nuclear systems are novel, with no similar study having been published before. 

Finally, an exploration of energy domain discretisation, or `group structure optimisation' was undertaken. How the energy domain is subdivided for nuclear analyses can have a significant effect on results. Previous methods for optimisation were mentioned, along with a discussion of nuclear resonant behaviour and the phenomenon of self-shielding. Having developed a framework for the generation of nuclear data on an arbitrary group structure, a method for the targeting of bin density is devised. The method starts with a logarithmically-spaced group structure and determines where self-shielding modifications to cross-sections most impact reaction rates. Distributions of effective self-shielding in energy are computed and summed for chosen nuclides. This distribution is then used as the basis for a bin density distribution and the apportioning of the energy domain. The method is applied to optimise group structures for two cases, a simple tungsten example and a more general group, optimised for a variety of metals of importance to fusion. Both conferred a significant advantage over traditional group structures in common usage today. The optimised 280 bin group structure was used to determine reaction rates in JET activation foils more accurately than the CCFE 709 bin group structure. Future work could involve the testing of optimised general group structures, to see if advantages are still conferred when optimisation is for a very large population of nuclides. Iterative applications of the algorithm were investigated during the course of this work, but did not confer any advantage of single applications. One alternative, related route for optimisation could be using the effective self-shielding distributions as defined in this work, but starting from hyperfine groups and removing the least necessary bounds until a target is reached. This approach may waste fewer bins than going from relatively coarse to locally fine, as is done in the current implementation.

Nuclear analyses are subject to a variety of sources of uncertainty. These can be from nuclear data, modelling approximations, discretisation of variables, amongst many other factors. Often these sources are simultaneously present in problems. \citeauthor{El-Guebaly2009}'s study of contributions to TBR uncertainty in lithium-lead blankets asserts that 90\% of the TBR margin required is to account for uncertainty in its estimation, only the remaining fraction is the required net gain for system losses \cite{El-Guebaly2009}. The uncertainty comes from both nuclear data (60\%) and modelling (30\%). Sometimes modelling approximations are synergistic in effect, as \citeauthor{Pelloni1989} noted that failing to account for self-shielding effects in breeding calculations is particularly important with homogenised geometries \cite{Pelloni1989}. 

Minimising uncertainty will become more important as we move towards constructing the first demonstration fusion devices, as unexpected or poor performance may reduce the momentum of these projects. We can continue to reduce sources of uncertainty in nuclear analyses through the development of improved methods, such as the energy group structure optimisation presented here. If approximations must be made for issues of limited computation, as with spatial homogenisation, then further analysis of the effects will be necessary to ensure their effects are adequately understood. Lastly, quantifying parameter distributions through the application of uncertainty propagation techniques like TMC will likely see an increased role--expanding from solely ND to sample other sorts of input distributions, perhaps including manufacturing tolerances and other parameters.








