%!TEX root = ../thesis.tex
%*******************************************************************************
%**************************** Introductory Chapter *****************************
%*******************************************************************************

\chapter{Introduction} %Title of first chapter

\ifpdf
    \graphicspath{{Chapter0/Figs/Raster/}{Chapter0/Figs/PDF/}{Chapter0/Figs/}}
\else
    \graphicspath{{Chapter0/Figs/Vector/}{Chapter0/Figs/}}
\fi

%********************************** %First Section  **************************************
\section{Background}
Hominids have always manipulated and altered their environment. Ancient people replanted wild flora and helped drive various megafauna such as woolly mammoth and steppe bison to extinction \cite{Mann2015} \cite{Pushkina2008}. Today, we are still changing our surroundings, but at a historically unprecedented rate. Paul Crutzen popularised the term anthropocene in the early 2000s, defining it as the current geological period of Earth's existence, one where humanity has a significant on the planet's various systems, in many ways outcompeting natural processes \cite{Crutzen2006}. Our rapacious desire for resources has felled forests, excavated vast holes in the Earth, driven countless species extinct and burnt fossil fuels at tremendous scale, releasing greenhouse gases (GHG) including carbon dioxide into the atmosphere. This last process has driven atmospheric CO$_{2}$ concentrations from 280ppm to 410ppm in two centuries \cite{}. The rate of accumulation is currently 2ppm and accelerating \cite{}. This accumulation of GHGs is driving climate change, leading to an atmospheric temperature increase of X since Y \cite{}. The consequences of the increased energy retained in the Earth system are many and varied. Sea levels have risen by X in Y years \cite{}. Arctic sea ice coverage has receeded by X since Y \cite{}. Oceans have warmed and acidified, resulting in coral loss of X \cite{}. The condition of the planet will undoubtedly continue to deteriorate in the coming decades. It is highly likely that increasingly frequent droughts will engender conflict \cite{}. Tropical diseases will spread from equatorial regions to the warming temperate zones \cite{}. Extreme weather events will probably become more frequent \cite{}. These conditions are now `baked-in' and will happen regardless of humanities efforts to avert them. This is because there is a long lag between addition of GHG to the atmosphere and a new temperature equilibrium being reached \cite{}. Two degrees of atmospheric warming over pre-industrial temperature is now extremely likely \cite{}. We must now mitigate the risks posed by our changed and changing planet and act to prevent warming worse than two degrees, for the consequences of further increases are dire \cite{}.

% Reference David McKay? Check his figures for consumption and correct if necessary
The main sources of GHG emission that humans are responsible for are combustion of fossil fuels for heat (industrial and domestic heating and cement production), transport (internal combustion engines and jet engines) and electricity production (typically with a boiler and steam turbine) and agriculture (deforestation and animals' methane production) \cite{}. We must rapidly reduce our GHG emission from current levels to near-zero or even negative rates \cite{}. In order to achieve this, heating, transport and electricity production must be decarbonised. Unfortunately, removing fossil fuels from heating and transport is most likely to be achieved with their electrification. This, coupled with a growing population and desire for increasing material living standards, means low-carbon electricity demand is going to dramatically increase in the near future \cite{}. 

%\cite{Zalasiewicz2017}

Low-carbon electricity production is currently produced from nuclear fission and renewables such as solar photovoltaic, solar thermal, wind, hydro, tidal, geothermal and biofuels. The acronym WWS (Wind, Water and Solar) summarises the renewable methods with greatest potential for widespread adoption. 

While nuclear fission is very safe by the metric of energy produced per death caused \cite{} it has spawned several major accidents, suffers from issues of public acceptability, radioactive waste storage and, more recently, construction cost. Additionally, the current trend for an open fuel cycle where nuclear fuel is not reprocessed and then reused is wasteful and drastically limits the potential of nuclear fission at scale \cite{}. WWS have tremendous potential and are currently gaining market share, with costs falling rapidly. For instance, some utilities are now bidding to supply electricity for as little as 1.2p per $kW \cdot h$ \cite{}. For comparison, the Hinkley Point C nuclear fission plant will have a minimum supply price of 9.25p per $kW \cdot h$ \cite{}. Unfortunately, intermittancy is a significant problem for WWS, specifically wind and solar. Much work is required to fully exploit its potential: drastic improvements in energy storage methods, a commensurate deployment of such technology, intelligent demand optimisation and improved distribution networks. Even then, it is unclear if WWS and/or nuclear fission could provide enough electricity for our future society \cite{}. A potential future alternative to both WWS and nuclear fission is nuclear fusion, the process of combining nuclei by which stars shine.

% Rebuttal to those Californian 100% WWS papers? 

\section{Nuclear fusion}
...

\subsection{Possible reactions}

\subsection{Confinement approaches}

\subsection{Progress to date (2018)}

\section{Neutron-matter interactions}

\subsection{?}

\section{Sources of uncertainty in fusion neutronics}
The sources of uncertainty in fusion neutronics are many and varied. Currently envisioned fusion reactors such as ITER rank as some of the most complex machines ever. Their multitude of components, diversity of materials and range of scales make simulating their operation challenging. While the devices themselves are large and difficult to model, introducing uncertainties into calculations, other kinds of uncertainty are inherent in the practice of neutronics more generally. Knowledge of nuclear physics, colloquially referred to as `nuclear data' (ND) is incomplete and uncertain for a variety of reasons. These include:  

\section{Implications of current uncertainties}

\section{Thesis outline}
