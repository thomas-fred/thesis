%!TEX root = ../thesis.tex
%*******************************************************************************
%**************************** Introductory Chapter *****************************
%*******************************************************************************

\chapter{Introduction} %Title of first chapter

\ifpdf
    \graphicspath{{Chapter0/Figs/Raster/}{Chapter0/Figs/PDF/}{Chapter0/Figs/}}
\else
    \graphicspath{{Chapter0/Figs/Vector/}{Chapter0/Figs/}}
\fi

%********************************** %First Section  **************************************
\section{Motivation}
Hominids have always manipulated and altered their environment. Ancient people replanted wild flora and helped drive various megafauna such as woolly mammoth and steppe bison to extinction \cite{Mann2015} \cite{Pushkina2008}. Today, we are still changing our surroundings, but at a historically unprecedented rate. Paul Crutzen popularised the term anthropocene in the early 2000s, defining it as the current geological period of Earth's existence, one where humanity has a significant on the planet's various systems, in many ways outcompeting natural processes \cite{Crutzen2006}. Our rapacious desire for resources has felled forests, excavated vast holes in the Earth, driven countless species extinct and burnt fossil fuels at tremendous scale, releasing greenhouse gases (GHG) including carbon dioxide into the atmosphere. This last process has driven atmospheric CO$_{2}$ concentrations from 280ppm to 410ppm in two centuries \cite{}. The rate of accumulation is currently 2ppm and accelerating \cite{}. This accumulation of GHGs is driving climate change, leading to an atmospheric temperature increase of X since Y \cite{}. The consequences of the increased energy retained in the Earth system are many and varied. Sea levels have risen by X in Y years \cite{}. Arctic sea ice coverage has receeded by X since Y \cite{}. Oceans have warmed and acidified, resulting in coral loss of X \cite{}. The condition of the planet will undoubtedly continue to deteriorate in the coming decades. It is highly likely that increasingly frequent droughts will engender conflict \cite{}. Tropical diseases will spread from equatorial regions to the warming temperate zones \cite{}. Extreme weather events will probably become more frequent \cite{}. These conditions are now `baked-in' and will happen regardless of humanities efforts to avert them. This is because there is a long lag between addition of GHG to the atmosphere and a new temperature equilibrium being reached \cite{}. Two degrees of atmospheric warming over pre-industrial temperature is now extremely likely \cite{}. We must now mitigate the risks posed by our changed and changing planet and act to prevent warming worse than two degrees, for the consequences of further increases are dire \cite{}.

% Reference David McKay? Check his figures for consumption and correct if necessary
The main sources of GHG emission that humans are responsible for are combustion of fossil fuels for heat (industrial and domestic heating and cement production), transport (internal combustion engines and jet engines) and electricity production (typically with a boiler and steam turbine) and agriculture (deforestation and animals' methane production) \cite{}. We must rapidly reduce our GHG emission from current levels to near-zero or even negative rates \cite{}. In order to achieve this, heating, transport and electricity production must be decarbonised. Unfortunately, removing fossil fuels from heating and transport is most likely to be achieved with their electrification. This, coupled with a growing population and desire for increasing material living standards, means low-carbon electricity demand is going to dramatically increase in the near future \cite{}. 

%\cite{Zalasiewicz2017}

Low-carbon electricity production is currently produced from nuclear fission and renewables such as solar photovoltaic, solar thermal, wind, hydro, tidal, geothermal and biofuels. The acronym WWS (Wind, Water and Solar) summarises the renewable methods with greatest potential for widespread adoption. 

While nuclear fission is very safe by the metric of energy produced per death caused \cite{} it has spawned several major accidents, suffers from issues of public acceptability, radioactive waste storage and, more recently, construction cost. Additionally, the current trend for an open fuel cycle where nuclear fuel is not reprocessed and then reused is wasteful and drastically limits the potential of nuclear fission at scale \cite{}. WWS have tremendous potential and are currently gaining market share, with costs falling rapidly. For instance, some utilities are now bidding to supply electricity for as little as 1.2p per $kW \cdot h$ \cite{}. For comparison, the Hinkley Point C nuclear fission plant will have a minimum supply price of 9.25p per $kW \cdot h$ \cite{}. Unfortunately, intermittancy is a significant problem for WWS, specifically wind and solar. Much work is required to fully exploit its potential: drastic improvements in energy storage methods, a commensurate deployment of such technology, intelligent demand optimisation and improved distribution networks. Even then, it is unclear if WWS and/or nuclear fission could provide enough electricity for our future society \cite{}. 

% Rebuttal to those Californian 100% WWS papers? 

\section{Nuclear fusion}
A potential future alternative to both WWS and nuclear fission is nuclear fusion, the process of combining nuclei by which stars shine. Nuclear fusion power has several desirable attributes:

\begin{itemize}
  \item Low CO\textsubscript{2} emissions
  \item High fuel energy density (reduced resource extraction burden)
  \item Abundant fuel (or feedstock for the manufacture of fuel)
  \item Dispatchable power generation
  \item No criticality (runaway) risk, unlike nuclear fission power
  \item High plant power density (GW scale facilities on the order of 1 km\textsuperscript{2}). 
\end{itemize}

However, to date the goal of electricity generation from nuclear fusion has eluded scientists and engineers.  

\subsection{Discovery}
Einstein's energy-mass equivalence, $E=mc^{2}$ led Arthur Eddington to hypothesise that the energy released by the sun was through the fusion of hydrogen nuclei into helium nuclei. However, contemporary predictions of the temperature required for the sun's power output did not match observations. Indeed, the temperatures observed seemed too low for nuclei to overcome the Coulomb energy barrier - the repulsive force experienced by two nuclei. The development of a theory of quantum tunnelling in the late 1920s by Gamow and others, predicted thermonuclear temperatures in accordance with observations of the sun and resolved the problem of stellar temperatures.

In the 1930s Mark Oliphant, Paul Harteck and Ernest Rutherford conducted particle beam experiments, bombarding various targets with deuterons. After several less interesting combinations, they observed ``on bombarding heavy hydrogen with diplons\footnote{Diplon was a contemporary name for what we now call a deuteron.} an enormous effect was produced'' \cite{oliphant1934}. This included the emission of high energy neutrons. Soon thereafter, plans for fusion weapons and controlled fusion power were developed.

%https://www.chemteam.info/Chem-History/Rutherford-1934b/Rutherford-1934b.html

\subsection{Possible reactions to harness}
Trying to fuse nuclei is difficult, as scattering via the Coulomb force is far more likely than fusion for all potential reactants. This Coulomb barrier is proportional to the product of the charges of the reactants as shown in equation \ref{eq:coulomb}, where $k_{e}$ is the Coulomb constant, $Z_{i}$ the respective atomic numbers, $e$ the charge on the electron and $r$, the interaction radius.

\begin{equation}
  \label{eq:coulomb}
  U_{coul} = k_{e}\frac{Z_{1}Z_{2}e^{2}}{r}
\end{equation}

Given this, it is only plausible to employ light nuclei for energy production. Of the light nuclei, certain reactions are more likely than others, with a greater cross-section for a given collision energy. They are shown in table \ref{tab:fuels} below.

\begin{table}[h]
\centering
\caption{Comparison of potential fusion fuels. The Q-value is the combined kinetic energy of the reaction products. \sigma is the cross-section of the reaction at a given energy, with centre-of-mass energies.}
\label{tab:fuels}
\begin{tabular}{@{}lllll@{}}
\toprule
Fuel          & Products          & Q-value (MeV) & \sigma \mathrm{(10 keV) (barn)} & \sigma \mathrm{(100 keV) (barn)} \\ \midrule
D + T         & \alpha + n        & 17.6          & $2.72\times10^{-2}$    & 3.43                    \\
D + D         & T + p             & 4.04          & $2.81\times10^{-4}$    & $3.3\times10^{-2}$      \\
D + D         & $^{3}$He + n      & 3.27          & $2.78\times10^{-4}$    & $3.7\times10^{-2}$      \\
D + $^{3}$He  & \alpha + p        & 18.3          & $2.2\times10^{-7}$     & 0.1                     \\
p + $^{11}$B  & 3\alpha           & 8.68          & $4.6\times10^{-17}$    & $3\times10^{-4}$        \\ \bottomrule
\end{tabular}
\end{table}

At an energy of 10 keV the D-T fusion reaction has a likelihood 100-fold of the next easiest reactions. It also has a large Q-value, with 17.6 MeV of binding energy released by the reconfiguration of the fuel nucleons. This energy is split between the products such that they have equal momentum, with the \alpha-particle receiving $1/5 17.6 \mathrm{MeV} = 3.5 \mathrm{MeV}$ and the neutron the remaining 14.1 MeV. D-$^{3}$He and p-$^{11}$B fuels are aneutronic, with only charged products. These fuels would obviate several difficult problems associated with D-T fusion which arise from the energetic emitted neutrons. The principal benefit would be a reduced burden on materials science to develop radiation tolerant structural and plasma facing materials, qualified in time for usage in a reactor. There would also be a significant reduction in the quantity of radioactive waste generated, compared with a D-T fuelled reactor. Unfortunately, aneutronic fuels require extraordinary temperatures to fuse, far beyond anything achieved to date, or conceived as possible in a tokamak in the near future. D-T fuel is currently envisioned as the fuel mixture for nearly all government sponsored research programmes. Some private efforts are attempting to use other fuels, especially those aforementioned aneutronic reactions.

\subsection{Development}
The 1950s saw the invention and development of so-called `pinch` fusion concepts. As a current flows through a tube of conducting medium (such as a plasma) the magnetic field generated acts in tandem with the current to crush the material to a filament by a Lorentz, or $\mathbf{j \times B}$ force. This is a kind of confinement, momentarily holding plasma ions in an increased density state. By bending a tube into a toroidal shape, matter might be circulated about a device, permanently confined. In reality, the gradient of the electric field across the torus acts to break confinement and lose material. The pinch concept was developed by Soviet scientists Sakharov and Tamm, into the tokamak. This was similar to previous designs of toroidal pinch machines, such as the British ZETA, but the relative strength of fields was different. Rather than having the dominant field be produced by the plasma current, in a tokamak the dominant field is generated by the toroidal field coils which wrap around the toroidal plasma chamber. Over the coming decades this would prove to be a more stable confinement system than other approaches. 

Throughout the rest of the 20th century, the tokamak concept came to dominate magnetic confinement fusion research. Plasma physicists around the world continued to design devices and plasma scenarios which attempt to avoid plasma instabilities and maximise the plasma triple product, developed from Lawson's criterion \cite{lawson1955}. The triple product, $nT\tau_{E}$, is the product of the plasma density, temperature and energy confinement time, respectively. Figure \ref{fig:triple_product} shows a plot of this performance metric against time. One can see the progress in relation to the well known Moore's Law in computing performance per unit cost and the beam energy of various particle colliders. Viewed in this way, fusion power has achieved remarkable progress across the decades (logarithmic and temporal).

\begin{figure}
%  \figuretitle{}
  \centering
  \includegraphics[width=0.8\textwidth]{triple_product.png}
  \caption{Fusion triple product achieved plotted against time. Similar progress measures for computing and particle physics are also shown. }
  \label{fig:triple_product}
\end{figure}

\subsection{Future plans}
There are many aspects of nuclear fusion power which require significant work before an electricity producing reactor could be constructed. In the field of plasma physics these include, improving energy and particle confinement, quelling various powerful emissions from the plasma (disruptions, ELMs, fast ions, runaway electrons) and increasing plasma temperatures. The ITER reactor under construction in the south of France will provide a test for plasma physics knowledge, through the equilibrium operation of a `burning' plasma. However, the plasma `core' is only one aspect of a functioning fusion reactor. The surrounding systems of an electricity producing reactor necessarily include:

\begin{itemize}
  \item Fuelling by gas or pellet injection
  \item Diagnostics to determine the state of the plasma and the plant
  \item Heating systems including micro and radiowaves and energetic particle beams
  \item Plasma heat and particle exhaust (divertor)
  \item Blanket for fuel breeding and capture of neutron energy
  \item Electricity production with heat exchangers, gas turbines and generators
  \item Tritium extraction, storage and recycling
  \item Cryoplant for the cooling of magnetic coils and other systems
  \item Remote handling for the maintenance of the other systems
\end{itemize}

While ITER will advance our understanding and experience of several of these ancillary systems, additional devices will be required before fusion electricity can be realised. These might include one or several demonstration reactors. Progress is crucial in materials science, as the systems listed above must operate for years without being degraded by interactions with extreme temperatures, plasma etching and radiation. Consequently, it is commonly help that a device for the production of 14.1 MeV neutrons at a useful flux is also required for the development and testing of radiation tolerant materials.

\section{Radiation-matter interactions}
Nuclear fusion gives rise to various energetic particles. The neutrons are emitted with $\frac{4}{5}$ of the total energy liberated, to give 14.1 MeV of kinetic energy. Photons are constantly generated in a tokamak plasma. Low energy photons are created through the excitation and de-excitation of atomic electrons, while Bremsstrahlung radiation arised from the acceleration of charged particles. This mixed radiation field is then further complicated by numerous interaction process, as neutrons and photons from the plasma interact with the device. What follows is a short primer on the possible interaction processes for the principle particles, the neutron and photon.

\subsection{Neutron}
The neutron is a sub-atomic particle of mass $1.674927471(21)\times10^{-27}$kg \cite{}, composed of three quarks. It can exist as part of a nucleus or unbound, where it has a mean lifetime of $881.5(15)$s \cite{}. While it is itself uncharged it can produce ionising radiation through interactions with other matter. These reactions are via the strong, weak, graviational and electromagnetic\footnote{While the neutron has zero electric charge, it does have a magnetic moment, and is therefore acted upon by electromagnetic fields.} forces. 

When neutrons undergo collisions, the ensuing process is strongly dependent on the combined energy of the reactants. Shown below is a diagram of the relevant processes across the energy domain.

... fig ...

The neutron-matter interaction processes of interest for this work are outlined below.

\subsubsection{Elastic scattering}
Neutrons can scatter off nuclei. In the context of fusion neutronics an elastic scattering event is defined as a neutron-nucleus reaction where no kinetic energy is transferred into excitation of the nucleus. Both the kinetic energy and momentum of the reactants are conserved. Although scattering is a quantum-mechanical phenomenon, properly described by the interacting wave functions, in this case a 'billiard ball' treatment satisfactorily describes reactions. 

The fast fusion ($> 1 \mathrm{MeV}$) neutrons are significantly more energetic than the nuclei in condensed matter ($\circa 1 \mathrm{eV}$), as such, the target nuclei may be thought of as at rest. In elastic scattering the energy lost by an incident neutron is gained by the target nucleus. The energy transferred is a function of the target nucleus mass, as shown below.

\begin{equation}
  \frac{1}{2}m_{n}v_{n,i}^{2} = \frac{1}{2}m_{n}v_{n,f}^{2} + \frac{1}{2}m_{a}v_{a,f}^{2}
  \label{eq:conserve_energy}
\end{equation}

Where $m_{n}$ is the mass of the neutron, $m_{a}$ the mass of the target nucleus, $v_{n,i}$ is the initial velocity of the neutron, $v_{n,f}$ the final velocity of the neutron and $v_{a,f}$ the final velocity of the nucleus. Conservation of momentum can be written as below.

% Conservation of energy, equation \ref{eq:conserve_energy} may be rewritten as the following.
%
% \begin{equation}
% %  v_{n,i}^{2} - v_{n,f}^{2} = \frac{m_{a}}{m_{n}}v_{a,f}^{2}
%   (v_{n,i} - v_{n,f})(v_{n,i} + v_{n,f}) = \frac{m_{a}}{m_{n}}v_{a,f}^{2}
%   \label{eq:conserve_energy_2}
% \end{equation}
%
% Given conservation of momentum the below also applies.
%

\begin{equation}
  m_{n}v_{n,i} = m_{a}v_{a,f} - m_{n}v_{n,f}
  \label{eq:conserve_momentum}
\end{equation}

With equations \ref{eq:conserve_energy} and \ref{eq:conserve_momentum} one can derive a quantity known as $\alpha$, the fraction of the initial energy retained by a neutron in a head on collision, as a function of target nucleus mass, A. For the derivation see \cite{harms1975}. The expression for $\alpha$ is as follows.

\begin{equation}
  \alpha(A) = (\frac{1-A}{1+A})^2
  \label{eq:alpha}
\end{equation}

One can see that lighter nuclei, of closer mass to neutrons more effectively moderate the energy of neutrons. A collision with a single proton will halt the incident neutron, $\alpha(1) = 0$. Heavier nuclei such as $^{186}$W give $\alpha(186) = 0.9787$ so scattered neutrons retain almost all of their energy.

%
% Equivalently expressed as the following.
%
% \begin{equation}
%   v_{n,i} + v_{n,f} = \frac{m_{a}}{m_{n}}v_{a,f}
%   \label{eq:conserve_momentum_2}
% \end{equation}
%
% Dividing equation \ref{eq:conserve_energy_2} by equation \ref{eq:conserve_momentum_2} results in the speed of the nucleus after the collision.
%
% \begin{equation}
%   v_{a,f} = v_{n,i} - v{n,f}
%   \label{}
% \end{equation}

\subsubsection{Nuclear reactions}
Nuclear reactions are distinct from elastic scattering. Rather than solely redistributing energy and momentum, reactions reconfigure nuclei and create new particles. There are two principle ways this can happen:

\begin{itemize}
  \item Direct nuclear reactions - a single nucleon in the target particle interacts with the incident particle. The interaction time is very limited, around $10^{-21}\mathrm{s}$ (i.e. the incident energy is high). Direct reaction products are anisotropically distributed, typically forward focused.
  \item Compound nucleus reactions - many nucleons interact together over a greater period of time, perhaps $10^{-18} - 10^{-16}\mathrm{s}$. The incident particle and the target nucleus coalesce into a new, excited nucleus. The collection of nucleons within the nucleus reach thermal equilibrium after a series of collisions. At some point, the excited compound nucleus will decay, but the mode of decay will not depend on the method of compound nucleus formation. Instead, the decay mode, or `exit channel' is dependent on the compound nucleus excitation energy and various nucleus-specific probabilities. 
\end{itemize} 

This dichotomy is not entirely accurate, as there is always some multiple internuclear interaction, but for so-called direct nuclear reactions, this is minor. Reactions can also be exothermic or endothermic, liberating or requiring energy to occur, respectively. This is recorded by the `Q-value' of a given reaction, say $^{208}$Pb(n,2n)$^{209}$Pb, where Q = -7.37 MeV. This neutron multiplying reaction is endothermic and requires an energy input to occur. A plot of cross-section against energy will clearly show this `threshold' behaviour, with zero probability of reaction until the minimum input energy is achieved. Exothermic reactions, by contrast can occur at any energy 

% I need something on compound nuclei as resonances. Interaction cross-sections should have been introduced prior to this.

\paragraph{Inelastic scattering}
An inelastic neutron scatter involves the target nucleus absorbing the incident neutron, forming an excited, compound nucleus and then re-emitting the neutron. The nucleus remains excited and emits a gamma ray to dissipate excess energy and reach its ground state. Given this and radiative capture, neutron fields usually beget gamma fields.

\paragraph{Radiative capture}
Radiative capture is similar to inelastic scattering, with an incident neutron also forming an excited compound nucleus. However, here a neutron is not re-emitted, only a gamma ray. This reaction is more likely at lower energies, given the slower relative velocity of reactants and therefore longer period for interaction. The reaction acts as a neutron `sink', removing neutrons from the system and is therefore an essential component of neutron shielding, along with moderating material and gamma-shielding.

\paragraph{Fission}
Inter-nucleon nuclear forces hold nucleons together in a spherical nucleus. With the addition of an extra nucleon (and its kinetic energy), this attraction can be overcome, distorting the nucleus into a dumbell shape, with two lobes. Sufficient energy will mean these lobes repel beyond the point of no return. As the electrostatic repulsion of the positively charged particles becomes greater in magnitude than the attractive nuclear force, the nucleus is fissioned. Nuclear fission is not normally associated with nuclear fusion reactors, however the fissioning of lithium in breeder blankets is an important process for the viability of fusion as an energy source. Uranium is present in beryllium containing ores and even after refining, some remains as an impurity. One variant of the proposed European fusion reactor design, DEMO is set to contain 560 t of beryllium. At 30 wppm uranium concentration (the ITER requirement) this results in 17 kg of uranium\cite{kolbasov2016}. A small amount is fissile as $^{235}$U, while the remaining $^{238}$U is fissionable by high energy neutrons. The daughter nuclei from actinide fission undoubtedly complicate the process of reactor component recycling\cite{cambi2010}.

\paragraph{Multiplication}
An incident neutron can result in multiple neutrons being emitted, a so-called multiplication reaction, (n,2n), (n,3n), etc. the reaction is always endothermic. Multiplication reactions are important in the context of tritium breeding in nuclear fusion blankets, increasing the neutron flux and therefore the production of tritium. Higher yield multiplication reactions have a greater energy requirement and thus a higher energy threshold. Materials such as beryllium and lead have high multiplication cross-sections, with beryllium having the lowest energy threshold for (n,2n) by some margin at approximately 2.7 MeV.

\subsection{Photon}
The photon is an elementary particle, indivisible by nature. Aggregated, photons provide one way of conceptualising an electromagnetic field. Photons travel in a vacuum at 299,792,458 ms$^{-1}$ and are created by various processes, across a wide spectrum of energies. The sources inside a fusion reactor include various plasma processes, neutron-matter interactions and radioactive decay amongst others. 

\subsubsection{Radioactive decay}
The gamma ray, or gamma photon, is a photon emitted by a nucleus. They can have energies from several keV to many MeV, overlapping in energy with X-rays\footnote{X-rays are produced by interactions with between and with electrons. These processes include bremsstrahlung radiation, or X-ray fluorescence.}. 
Aside from an arbitrary energy and associated wavelength distinction, gamma photons and X-ray photons cannot be distinguished without knowledge of their source. The gamma ray is typically associated with...     Gamma photons are ionising radiation and thus biologically hazardous. 

\subsubsection{Bremsstrahlung}

\subsubsection{Photoelectric}

\subsubsection{Compton scattering}

\subsubsection{Pair production}

\section{Sources of uncertainty in fusion neutronics}
The sources of uncertainty in fusion neutronics are many and varied. Currently envisioned fusion reactors such as ITER rank as some of the most complex machines ever. Their multitude of components, diversity of materials and range of scales make simulating their operation challenging. While the devices themselves are large and difficult to model, introducing uncertainties into calculations, other kinds of uncertainty are inherent in the practice of neutronics more generally. Knowledge of nuclear physics, colloquially referred to as `nuclear data' (ND) is incomplete and uncertain for a variety of reasons. These include:  

\section{Implications of current uncertainties}

\section{Thesis outline}
