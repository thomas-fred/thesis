% ************************** Thesis Abstract *****************************
\begin{abstract}
This work explores the contribution of several sources of uncertainty to the neutronics of nuclear fusion devices. Background information, radiation-matter interactions and their descriptions as `nuclear data' are discussed before an introduction to several useful methods in nuclear analyses. After a discussion of the consequences of certain uncertainties in nuclear fusion systems, there is an analysis which quantifies aspects of uncertainty in tritium breeding. This analysis uses Total Monte Carlo sampling of nuclear parameters to determine the spread in Tritium Breeding Ratio (TBR) values for the proposed DEMO reactor, due to uncertainty in our knowledge of neutron interactions with the lead nucleus. The TBR is found to have a standard deviation (1 $\sigma$) of 1.2\% of the mean value. The higher-order distribution moments are also shown, including a low-value tail. Subsequently, modelling approximations in radiation shielding are investigated, specifically the practice of spatial homogenisation of radiation transport geometries. The effects of spatial homogenisation in radiation shielding for the ITER experiment are quantified for on-load and shut-down dose rates. These are found to contribute a relatively modest (22\% maximum deviation) change on the most likely value. More general conclusions about the applicability of spatial homogenisation as a technique are drawn. The thesis then explores how the energy domain is discretised in current analyses and whether an improved method, delivering greater accuracy in activation calculations, might be created. By first exploring self-shielding theory and distributions of nuclear resonances, a series of functions for assembling an optimum bin density are devised. A nuclear data processing workflow is developed to produce data on an arbitrary energy `group structure' and employed to create nuclear data libraries in an optimised group structure. These optimised discretisations are tested against and outperform current standard group structures in an example of relevance to the JET experiment. Recommendations for future work on improving the optimisation algorithm are given. Lastly, conclusions comparing the investigated contributions to analysis uncertainty are drawn.
\end{abstract}
